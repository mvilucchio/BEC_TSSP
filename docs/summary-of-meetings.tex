\documentclass[12pt, oneside]{article}   	
\usepackage{geometry}                		
\geometry{letterpaper}                   		
\usepackage{graphicx}			

\usepackage{amsmath}
\usepackage{amsfonts}
\usepackage{amssymb}
\usepackage{mathrsfs}
\usepackage{dsfont}

\usepackage{physics}
\usepackage{siunitx}
\usepackage{slashed}

\usepackage{mathtools}

\renewcommand\thesection{Meeting \arabic{section}. Date:}

\title{Summary of Meetings}
\author{M. Vilucchio, A. Ouzzani, C. Lopetegui}
%\date{}

\begin{document}
\maketitle
\section{2020-11-18}

The method used is the following, where $\varepsilon$ is the time step:
\[
	e^{\frac{\varepsilon}{2} (\operatorname{A})} e^{\varepsilon \operatorname{B}}e^{\frac{\varepsilon}{2} (\operatorname{A})} = e^{\varepsilon (\operatorname{A} + \operatorname{B})} + \order{\varepsilon^3}
\]
in the actual coding one useful thing to do is making the successive application of the exponential of $\operatorname{A}$ as one single operation. Like
\[
	\underbracket{e^{\frac{\varepsilon}{2} (\operatorname{A})}} \underbracket{e^{\varepsilon \operatorname{B}} }\underbracket{e^{\frac{\varepsilon}{2} (\operatorname{A})} e^{\frac{\varepsilon}{2} (\operatorname{A})} }_{e^{\varepsilon \operatorname{A}}} \underbracket{e^{\varepsilon \operatorname{B}} } \underbracket{e^{\frac{\varepsilon}{2} (\operatorname{A})} e^{\frac{\varepsilon}{2} (\operatorname{A})} }_{e^{\varepsilon \operatorname{A}}} \underbracket{ e^{\varepsilon \operatorname{B}} } \underbracket{e^{\frac{\varepsilon}{2} (\operatorname{A})} }
\]

While using the fft in two dimensions one should always remember to split space in every direction in $2^n$ parts to utilise the complexity advantage of the fft implementation.

The definition of the width is, in every direction:
\[
	\sigma_x = \sqrt{ \big\langle \qty(x - \langle x\rangle)^2\big\rangle}\quad \text{ with } \quad\langle f \rangle = \int_{\mathbb{R}^2} f \abs{\psi} \dd[2]x
\]

The implementation, for the dimensions of the wave-function and the space matrix should be vectorial.

Then to check the code try the example 2.I .

\paragraph{Questions}
\begin{itemize}
	\item What is the error in the scheme for both the temporal and the spatial discretisation?
\end{itemize}

\end{document}  